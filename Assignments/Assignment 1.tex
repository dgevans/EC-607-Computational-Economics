% !TEX TS-program = xelatex
\documentclass{exam}

\usepackage{enumitem}
\usepackage{amsmath}
\usepackage{amsfonts}
\usepackage{hyperref}
\usepackage{booktabs}

\firstpageheader{Econ 607 - Computational}{Assignment 1 - Due 4/12 at 11:59pm }{Page \thepage\ of \numpages}
\firstpageheadrule

\begin{document}
Consider a univariate stochastic process $y_t$ which follows a first-order moving average representation
\begin{equation}
	y_t = \epsilon_t - \theta\epsilon_{t-1}
\end{equation}where $\{\epsilon_t\}$ is an i.i.d. process distributed $\mathcal N(0,1)$ and $\theta > 1$

\begin{enumerate}
 	\item Write a program to simulate this process.  Simulate 100 periods when $\theta =2$. (Draw $\epsilon_{-1}$ from $\mathcal N(0,1)$ and $\theta > 1$)
 	\item Argue that $\epsilon_t$ cannot be expressed as a linear combination of $y_{t-j}$ for $j\geq0$ where the sum of the squares of the weights is finite. \textbf{Hint:} Note that 
 	\begin{align*}
 		\epsilon_t &= y_t + \theta\epsilon_{t-1}\\
 				   &= y_t + \theta y_{t-1} +\theta^2 \epsilon_{t-2}\\
 				   &\vdots
 	\end{align*} 
 	if you keep repeating what happens to the coeficients on $y_{t-j}$?  We call this property non-invertability.
 	\item Represent (1) using as a state space system using our Kalman filter.  What are the matrices $A,C,$ and $G$?  How do they depend on $\theta$? \textbf{Hint:} $x_t$ will be a vector.
 	\item Assuming initial beliefs about $\epsilon_{t-1}$ are $\mathcal N(0,1)$.  Write a program to apply this kalman filter to an arbitray sequence $\{y_t\}_{t=0}^T$ for an arbitray $\theta$.
 	\item Write a program to compute the log-likelihood of sequence $\{y_t\}_{t=0}^T$ for an arbitray $\theta$. \textbf{Hint: }  Using the Kalman filter the uncoditional distribution of $y_0$ is $\mathcal N(G\hat x_0),\Omega_0)$.  The distribution of $y_t$ conditional on the past histories of $y^{t-1}=(y_{t-1},y_{t-2},\ldots,y_0)$ is $\mathcal N(G\hat x_t,\Omega_t)$.  The likelihood of a sequence of data $(y_T,\ldots,y_0)$ can then be constructed from the conditional likelihoods
 	\begin{equation}
 		f(y_T,\ldots,y_0) = f(y_T|y^{T-1})f(y_{T-1}|y^{T-2})\cdots f(y_1|y_0)f(y_0)
 	\end{equation}
 	and similarly for the log likelihood.
 	\item For your simulation in part 1. Plot the log likelihood as a function of $\theta$.  Where is it maximized?  What happens if the simulation length is 1000 periods?

 	\item  \textbf{Bonus:} Repeat the Kalman filter exercise for $\theta = 0.5$.  What happens to $\Sigma_t$ over time?  How does this relate to invertability (can we express $\epsilon_t$ as a linear combination of $y_{t-j}$)?
 \end{enumerate} 
\end{document} 
