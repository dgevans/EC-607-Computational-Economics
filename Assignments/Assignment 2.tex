% !TEX TS-program = xelatex
\documentclass{exam}

\usepackage{enumitem}
\usepackage{amsmath}
\usepackage{amsfonts}
\usepackage{hyperref}
\usepackage{booktabs}

\firstpageheader{Econ 607 - Computational}{Assignment 2}{Page \thepage\ of \numpages}
\firstpageheadrule

\begin{document}
Consider the following of the stochastic growth model, in its social planner version:
\[
	V(a,k) = \max_{c,h,i,k'} \theta\log(c)+(1-\theta)\log(1-h)+\beta\mathbb E[V(a',k')]
\]subject to
\begin{align*}
	c + i &= \exp(a) k^\alpha h^{1-\alpha}\\
	k' &= (1-\delta)k + i
\end{align*}
\begin{enumerate}
 	\item Begin by assuming that $a=0$ is deterministic.  The model then has the following parameters $\{\beta,\theta,\alpha,\delta\}$.  Set $\beta = 0.98$ (to target a quarterly frequency).  Solve for the steady state of this model and calibrate the model to match (i) average hours worked $h^* = 0.3$; (ii) capital share of income of 0.3; (iii) an investment to output ratio of $i^*/y^*=0.16$.
 	\item Write a program to solve for the perfect foresight transition path $\{k_t\}_{t=0}^\infty$.
 	\item Choose a grid over capital and write a computer program to solve for the optimal policies in the non-stochastic environment using Value Function Iteration.  Check that the steady state of your approximated policy rules corresponds to  the steady state you computed in 1. and the path you computed in step 2.
 	\item  Solve for the policy rules with the additional restriction $i\geq 0$ (repeat steps 2 and 3).  When do the policy rules differ?  
 	\item  Now introduce uncertainty.  Download the ``logtfp\_detrended.csv''\footnote{I used the TFP series from \url{http://www.frbsf.org/economic-research/indicators-data/total-factor-productivity-tfp/} which was then logged and detrended with an hp filter (this is somewhat questionable) with scale parameter 1600.} from canvas.  Estimate the following process for $a_t$:
 	\[
 		a_t = \rho a_{t-1} +\sigma \epsilon_t
 	\]with $\epsilon_t$ standard normal.  Approximate the AR(1) process using the Rouwenhorst method with $N=25$ grid points.\footnote{The QuantEcon package has a built in function to do this.}
 	\item  Start from the steady state and simulate for 10,000 a timeseries for output, consumption, investment and hours worked.  Drop the first 3,000 realizations.  On the remaining 7,000 compute the standard deviations, autocorrelations, and cross-corellations for all of these variables. (Don't impose that $i_t\geq0$).
 	\item  Repeat 6. with the additional restriction of $i\geq 0$.  Is there a difference?  Why or why not?  In what regions of the state space does the constraint bind?  How frequently are they visited?\footnote{You might find it helpful to simulate a variable which is one when the constraint is binding and zero otherwise.}
 \end{enumerate} 
\end{document} 
